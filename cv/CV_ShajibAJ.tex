\let\latexnofiles\nofiles
\let\nofiles\relax
\documentclass[margin, line]{res}

%\usepackage[margin=1in]{geometry}
\usepackage[breaklinks,colorlinks,citecolor=magenta,linkcolor=red,urlcolor=magenta]{hyperref}
\usepackage{eurosym}
\usepackage{etaremune}

\oddsidemargin -.5in
\evensidemargin -.5in
\textwidth=6.0in
\itemsep=0in
\parsep=0in

% if using pdflatex:
%\setlength{\pdfpagewidth}{\paperwidth}
%\setlength{\pdfpageheight}{\paperheight} 

\newenvironment{list1}{
  \begin{list}{\ding{113}}{%
      \setlength{\itemsep}{0in}
      \setlength{\parsep}{0in} \setlength{\parskip}{0in}
      \setlength{\topsep}{0in} \setlength{\partopsep}{0in} 
      \setlength{\leftmargin}{0.17in}}}{\end{list}}
\newenvironment{list2}{
  \begin{list}{$\bullet$}{%
      \setlength{\itemsep}{0in}
      \setlength{\parsep}{0in} \setlength{\parskip}{0in}
      \setlength{\topsep}{0in} \setlength{\partopsep}{0in} 
      \setlength{\leftmargin}{0.2in}}}{\end{list}}


\begin{document}

\name{Anowar J. Shajib \vspace*{.1in}}

\begin{resume}
\section{\sc Contact Information}
\vspace{.05in}
\begin{tabular}{@{}p{3in}p{3in}}                  
Department of Physics and Astronomy    & {\it Office:} Knudsen Hall 3-145T   \\     
University of California, Los Angeles  & {\it Phone:}  (213) 271-7056 \\     
430 Portola Plaza, Box 951547 				         & {\it E-mail:}  \href{mailto:ajshajib@astro.ucla.edu}{ajshajib@astro.ucla.edu} \\
Los Angeles, CA 90095 USA  & {\it Web:} \url{www.astro.ucla.edu/~ajshajib} \\   
\end{tabular}


\section{\sc Research Interests}
Gravitational Lensing, Observational Cosmology

\section{\sc Education}
{\bf University of California, Los Angeles}, USA\\
%{\em Department of Statistics} 
\vspace*{-.1in}
\begin{list1}
	\item[] Ph.D. Candidate, Astronomy, March 2017 (expected
  	graduation date: June 2020)
	\begin{list2}
		\vspace*{.05in}
		\item Dissertation Topic:  ``Shining light on the dark energy with time-delay cosmography'' 
		%\item Dissertation Topic:  ``Hierarchical Models for Multiple Ratings
		%  in Performance-Based\\ \hspace*{1.23in} Student Assessments.'' 
		\item Advisor:  Prof. Tommaso Treu
	\end{list2}
\end{list1}
\vspace*{.05in}

\begin{list1}
	\item[] M.S., Astronomy,  June 2016
	\begin{list2}
		\vspace*{.05in} 
		\item Advisor:  Prof. Edward L. Wright
	\end{list2}
\end{list1}

{\bf The University of Tokyo}, Japan\\
%{\em Department of Mathematics and Statistics} 
\vspace*{-.1in}
\begin{list1}
\item[] B.S., Physics,  March 2014
\end{list1}


\section{\sc Honors and Awards} 
Graduate Student Travel Stipend, {MIAPP}, 2018, \EUR{500}

\vspace*{-2.5mm}
Graduate Student Travel Grant, {UCLA}, 2017, \$2000

\vspace*{-2.5mm}
\textbf{Graduate Division Fellowship}, UCLA, 2014-2015, \$18,000

\vspace*{-2.5mm}
\textbf{MEXT\footnote{Ministry of Education, Culture, Sports, Science and Technology, Government of Japan} Scholarship}, 2009-2014 (equivalent to \$92,000)


\section{\sc Publications}
\textbf{First Author Publications} \\
\begin{etaremune}
	\item {Shajib, A. J.}, et al. Is every strong lens model unhappy in its own way? Uniform modelling of a sample of 13 quadruply+ imaged quasars. \href{https://doi.org/10.1093/mnras/sty3397}{MNRAS, 483, 5649-5671, 2019}.
	\item {Shajib, A. J.}, Treu, T., and Agnello, A. Improving time-delay cosmography with spatially resolved kinematics. \href{https://doi.org/10.1093/mnras/stx2302}{MNRAS, 473, 210-226, 2018}.
	\item {Shajib, A. J.} and Wright, E. L. Measurement of the integrated Sachs-Wolfe effect using the AllWISE data release. \href{http://dx.doi.org/10.3847/0004-637X/827/2/116}{ApJ, 827:116 (9pp), 2016}.
\end{etaremune}


\textbf{Contributing Author Publications} \\
\begin{etaremune}
	\item Beaton, R. L., et al. Measuring the Hubble Constant Near and Far in the Era of ELT's. \href{https://arxiv.org/abs/1903.05035}{arXiv:1903.05035, 2019}.
	\item Birrer, S., et al. H0LiCOW - IX. Cosmographic analysis of the doubly imaged quasar SDSS 1206+4332 and a new measurement of the Hubble constant. \href{https://doi.org/10.1093/mnras/stz200}{MNRAS, stz200, 2019}.
	\item Chen, G. C.-F., et al. Constraining the microlensing effect on time delays with new time-delay prediction model in $H_0$ measurements. \href{https://doi.org/10.1093/mnras/sty2350}{MNRAS, 481, 1115-1125, 2018}.
	\item Ding, X., Treu, T., {\bf Shajib, A. J.}, et al. Time Delay Lens Modeling Challenge: I. Experimental Design. \href{https://arxiv.org/abs/1801.01506}{arXiv:1801.01506, 2018}.
 	\item Williams, P. R., et al. Discovery of three strongly lensed quasars in the Sloan Digital Sky Survey. \href{https://doi.org/10.1093/mnrasl/sly043}{MNRAS: Letters, 477, L70-L74, 2018}.
	%\item Molina, E., et al. More massive galaxies are more massive: luminous and dark matter in small-separation quasar lenses. In preparation.
\end{etaremune}
%\section{\sc Papers in preparation}
%\textbf{Shajib, A.J.} et al. Improving time-delay cosmography with spatially resolved kinematics. Submitted to MNRAS, 2017.


\section{\sc Invited Talks}
\begin{etaremune}
	\item MPA Lensing Group Seminar, Munich, Germany, June 2018.
\end{etaremune}


\section{\sc Contributed Talks}
\begin{etaremune}
	\item Keck Science Meeting, Caltech, USA, September 2018.
	\item Extragalactic distance scale in the \textit{GAIA} era, MIAPP workshop, Munich, Germany, June 2018.
	\item Shedding Light on the Dark Universe with Extremely Large Telescopes, UCLA, USA, April 2018.
	\item Strong Lensing by Galaxies and Clusters, Aosta, Italy, June 2017.
\end{etaremune}


\section{\sc Professional Service}
\begin{itemize}
\item Journal referee for Monthly Notices of the Royal Astronomical Society and American Astronomical Society
\item Proposal reviewer for \textit{Hubble Space Telescope}
\item Graduate admission committee member (2019), Division of Astronomy, UCLA
\end{itemize}

\section{\sc Academic Experience}
{\bf University of California, Los Angeles}, USA

\vspace{-.3cm}
{\em Graduate Student} \hfill {\bf October 2014 - present}\\
Includes current Ph.D.~research, Ph.D.~and Masters level coursework and
research.


{\em Guest Lecturer} \hfill {\bf}\\
\begin{list2}
	\item Physics 127 - General Relativity (Spring 2015)
	\item Astro 81 - Astronomy I: Stars and Nebulae (Winter 2016)
\end{list2}
	
{\em Teaching Assistant} \hfill {}\\
\begin{list2}
	\item Astronomy 3 - Nature of Universe (Fall 2014)
	\item Physics 1C - Electrodynamics, Optics and Special Relativity (Winter 2015)
	\item Physics 127 - General Relativity (Spring 2015)
	\item Physics 6C - Physics for Life Sciences Majors: Light, Fluids, Thermodynamics, Modern Physics (Fall 2015)
	\item Astronomy 81 - Astrophysics I: Stars and Nebulae (Winter 2016)
	\item Astronomy 140 - Stellar Systems and Cosmology (Spring 2016)
	\item Physics 12 - Physics of Sustainable Energy (Winter 2017)
\end{list2}

{\bf European Southern Observatory}, Munich, Germany\\
%\vspace{-.1cm}
{\em Visiting Graduate Student} \hfill {\bf July 2018}\\
Collaborative research with Dr. Adriano Agnello.

%\vspace{-.1cm}
%{\em Instructor} \hfill {\bf May - June, 2002}\\
%Co-taught graduate level course for the Master of Science in
%Computational Finance program.  Shared responsibility for lectures, exams,
%homework assignments, and  grades.  
%\vspace*{.05in}  
%\begin{list2}
%\item 46-731 Probability and Statistics, Summer 2002.
%\end{list2}


%\vspace{-.1cm}
%{\em NSF VIGRE Teaching Fellow} \hfill {\bf January - May, 2001}\\
%Head teaching assistant.   
%Duties included  shared administrative responsibilities with faculty
%instructor, fielding of all student inquiries, and oversight of
%graduate student teaching assistants and graders.
%\vspace*{.05in}  
%\begin{list2}
%\item 36-217 Probability Theory and Random Processes, Spring 2001.
%\end{list2}


%\vspace{-.1cm}


\section{\sc Workshops}
\begin{etaremune}
	\item TMT Early Career Initiative Workshop, Los Angeles, December 2018.
	\item Extragalactic distance scale in the \textit{GAIA} era, MIAPP, Germany, June-July 2018.
	\item Mary Lea \& C. Donald Shane Observational Astronomy Workshop, UCO/Lick Observatory, October 2014.
\end{etaremune}


%Paciorek, C.J., J.S. Risbey, V. Ventura, and R.D.Rosen.  2001.  Changes in Northern Hemisphere winter storm activity (1949-1999) based
%on a comparison of cyclone indices.  8th International Meeting on
%Statistical Climatology, Luneberg, Germany, March, 2001.
%
%Paciorek, C.J. and R. Rosenfeld.  2000.  Minimum classification error
%training in exponential language models.  2000 Spring Transcription
%Workshop, College Park, Maryland.
%\vspace*{-.25in}  
%\begin{verbatim}http://www.nist.gov/speech/publications/tw00/html/abstract.htm#cp1-50\end{verbatim}

\section{\sc Mentoring}
\begin{itemize}
	\item \textbf{Eden Molina:} UCLA undergraduate, completing a project to model doubly-imaged lensed quasars from NIRC2 imaging data. Mentored since Fall 2018.
\end{itemize}

\section{\sc Outreach}
{\bf Cal-Bridge program}, hosted a workshop at UCLA for California State University undergraduates on Graduate admission preparation, March, 2019. \\
{\bf Lecturer at Astronomy Live! summer workshop} for high school students, 2018. \\
{\bf Astronomy Live!}, visited K-12 schools to perform various demos as part of the UCLA Astronomy outreach program. \\
{\bf Exploring Your Universe}, performed various demos in UCLA's annual science festival, 2014-17.
{\bf Star show}, UCLA Planetarium, 2014, 2015. \\
{\bf Public talk}, Title: The Story of You. UCLA Planetarium, 2014. \\

\section{\sc Approved Observing Proposals (CoI)}
\begin{etaremune}
\item \textit{Hubble Space Telescope} GO-15652 (2018). PI: Treu. $H_0$, the stellar initial mass function, and other dark matters from a large sample of quadruply imaged quasars (2018).
\item 2-m Himalayan Chandra Telescope (2018). PI: Courbin. Photometric monitoring of the quadruply lensed quasar PSOJ0147+4630.
\item MUSE NFM Science Verification (2018). PI: Zanella. From cosmology to star-forming regions: two compelling cases for MUSE NFM.
\item Keck U053(2017A), U032(2017B), U011(2018A),  U011(2018B), U029(2019A). PI: Treu. Dark energy with gravitational time-delay: OSIRIS spectroscopy of lensing galaxies.
\end{etaremune}


\section{\sc Observing Experience}
OSIRIS, Keck I, 11.5 nights,\\
NIRC2, Keck II, 3 nights. 


\section{\sc Data Analysis Experience}
\textit{Hubble Space Telescope} (WFC3), 
W. M. Keck Observatory (OSIRIS, NIRC2),
Very Large Telescope (MUSE),
{\it Wide-field Infrared Survey Explorer},
{\it Wilkinson Microwave Anisotropy Probe},
{\it Planck},
Sloan Digital Sky Survey.

%\section{\sc Professional Experience}
%{\bf Bureau of Transportation Statistics, U.S. Department of
%  Transportation}, Washington, District of Columbia USA
%
%\vspace{-.3cm}
%{\em Summer researcher} \hfill {\bf May, 2000 - August, 2000}\\
%Carried out several consulting projects, including modelling of
%injuries to cadavers in crash test experiments, analysis of airline
%delay data, and advice on analysis of airline economics data.
%
%{\bf Abt Associates}, Bethesda, Maryland USA
%
%\vspace{-.3cm}
%{\em Associate Programmer Analyst and Research Assistant} \hfill {\bf
%  October, 1994 - August, 1996}\\
%Researcher and computer model developer for U.S. EPA Regulatory Impact
%Analysis of Section 403 Lead Paint Hazard Rule.  Other projects
%included database analysis, literature reviews, and cost-benefit analysis.


\section{\sc Computer Skills} 
%\begin{list2}
\textbf{Programming Languages:} Python, C, C++, PHP, SQL, JavaScript \\
\textbf{Astronomy software:} IRAF, PyRAF, SExtractor, DS9, Lenstronomy \\
\textbf{Software/Framework:} TensorFlow, Flask
%\end{list2}


%\section{\sc Positions of Responsibility}
%{\bf Captain and Coach,} The University of Tokyo Cricket Club, 2012-13 \\
%{\bf College prefect,} Sylhet Cadet College, 2006-07


\end{resume}
\end{document}
