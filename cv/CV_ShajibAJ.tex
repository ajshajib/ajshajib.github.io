

\let\latexnofiles\nofiles
\let\nofiles\relax
\documentclass[margin, line]{res}

%\usepackage[margin=1in]{geometry}
\usepackage{fancyhdr}
\usepackage[breaklinks,colorlinks,citecolor=magenta,linkcolor=red,urlcolor=magenta]{hyperref}
\usepackage{eurosym}
\usepackage{etaremune}
\usepackage{fontawesome}
\usepackage{amsmath}
\usepackage{amssymb}

\oddsidemargin -.5in
\evensidemargin -.5in
\textwidth=6.0in
\itemsep=0in
\parsep=0in

\pagestyle{fancy}
% Clear the header and footer
\fancyhead{}
\fancyfoot{}
\renewcommand{\headrulewidth}{0pt}
% Set the right side of the footer to be the page number
\rfoot{\thepage}
\lfoot{\hspace*{-1.3in}Anowar J. Shajib: CV}

% if using pdflatex:
%\setlength{\pdfpagewidth}{\paperwidth}
%\setlength{\pdfpageheight}{\paperheight} 

\newenvironment{list1}{
  \begin{list}{\ding{113}}{%
      \setlength{\itemsep}{0in}
      \setlength{\parsep}{0in} \setlength{\parskip}{0in}
      \setlength{\topsep}{0in} \setlength{\partopsep}{0in} 
      \setlength{\leftmargin}{0.17in}}}{\end{list}}
\newenvironment{list2}{
  \begin{list}{$\bullet$}{%
      \setlength{\itemsep}{0in}
      \setlength{\parsep}{0in} \setlength{\parskip}{0in}
      \setlength{\topsep}{0in} \setlength{\partopsep}{0in} 
      \setlength{\leftmargin}{0.2in}}}{\end{list}}


\begin{document}

\name{Anowar J. Shajib \vspace*{.1in}}

\begin{resume}


\section{\sc Contact Information}
\vspace{.05in}
\begin{tabular}{@{}p{3in}p{3in}}                  
Department of Astronomy and Astrophysics  \\ % & {\it Office:} Knudsen Hall 3-145T   \\     
The University of Chicago  & {\it Phone:}  (213) 271-7056 \\     
5640 South Ellis Avenue 	         & {\it E-mail:}  \href{mailto:ajshajib@astro.ucla.edu}{ajshajib@uchicago.edu} \\
Chicago, IL 60637  & {\it Web:} \url{https://ajshajib.github.io} \\   
\end{tabular}


\section{\sc Research Interests}
Observational cosmology, galaxy evolution, strong gravitational lensing,  statistical methods, and machine learning

\vspace*{2.5mm}
\section{\sc Professional Positions}

\vspace*{.16in}
\begin{list1}
	\item[] \textbf{NHFP Einstein Fellow}, Department of Astronomy and Astrophysics, The University of Chicago, September 2021 -- present.
	\item[] \textbf{KICP Fellow}, Kavli Institute of Cosmological Physics, September 2021 -- present.
	\item[] \textbf{Postdoctoral Scholar}, Department of Astronomy and Astrophysics, The University of Chicago, November 2020 -- August 2021.
	%\item[] \textbf{Associate Fellow}, Kavli Institute of Cosmological Physics, November 2020 -- -- August 2021.
\end{list1}

\vspace*{-.05in}


\section{\sc Education}
{\bf University of California, Los Angeles}, USA\\
%{\em Department of Statistics} 
\vspace*{-.1in}
\begin{list1}
	\item[] Ph.D., Astronomy \& Astrophysics, September 2020.
	\begin{list2}
		\vspace*{.05in}
		\item Dissertation title:  ``The Hubble constant and The $\Lambda$CDM cosmology: A Magnified View using Strong Lensing'' 
		%\item Dissertation Topic:  ``Hierarchical Models for Multiple Ratings
		%  in Performance-Based\\ \hspace*{1.23in} Student Assessments.'' 
		\item Advisor:  Prof. Tommaso Treu
	\end{list2}
\end{list1}
\vspace*{.05in}

\begin{list1}
	\item[] M.S., Astronomy,  June 2016
	\begin{list2}
		\vspace*{.05in} 
		\item Advisor:  Prof. Edward L. Wright
	\end{list2}
\end{list1}

{\bf The University of Tokyo}, Japan\\
%{\em Department of Mathematics and Statistics} 
\vspace*{-.1in}
\begin{list1}
\item[] B.S., Physics,  March 2014
\end{list1}



\section{\sc Awards, Fellowships, and Honors}
\textbf{NASA Hubble Fellowship Program, Einstein Fellowship,} 2021 \\
\textbf{Rodger Doxsey Travel Prize} for dissertation talk, 235\textsuperscript{th} American Astronomical Society Meeting, Hawaii, USA, 2020 

\vspace*{-2.5mm}
\textbf{Dissertation Year Fellowship} from UCLA Graduate Division, 2019-2020%, \$20,000

\vspace*{-2.5mm}
Graduate Student Travel Stipend, {Munich Institute for Astro- and Particle Physics}, 2018%, \EUR{500}

%\vspace*{-2.5mm}
%Graduate Student Travel Grant, {UCLA}, 2017 %, \$2000

\vspace*{-2.5mm}
\textbf{Graduate Division Fellowship}, UCLA, 2014-2015%, \$18,000

\vspace*{-2.5mm}
\textbf{Full-tuition Undergraduate Scholarship} from {Government of Japan, Ministry of Education, Culture, Sports, Science, and Technology}, 2009-2014% (equivalent to \$92,000)

\section{\sc Publication statistics} 39 total published/submitted papers (listed at the end of CV). {11 first-author}, {4 second-author, and 24 other co-authored} papers. \textbf{First author citations}: 440, \textbf{total citations:} 3,268, h-index: 23 (10/15/2023, \href{https://ui.adsabs.harvard.edu/user/libraries/NYgiA71JS4CR85Tt8CgJsw}{ADS}).
\vspace*{2.5mm}

\section{\sc Approved Proposals \& Grants (PI/Co-PI)}
\begin{enumerate}
	\item \textit{JWST} GO-2974 (2023). The Hubble constant at 1.9\% from spatially resolved kinematics of gravitational lens.
	\item \textit{Hubble Space Telescope} Schedule Gap Program SNAP-17307 (2023). A Legacy Library of 500 Strong Gravitational Lenses.
	\item \textit{Hubble Space Telescope} AR-16149 (2020). Systematics in $H_0$ from lensing: a comprehensive study of internal structure in elliptical galaxies. \textbf{Grant: $\sim$\$100K}.
\end{enumerate}

\section{\sc Approved Computing Proposals (PI/Co-PI)}
\begin{enumerate}
\item UChicago Midway2 Research Allocation, 1,000,000 CPU hours. First semester, 2022--23.
\item UChicago Midway2 Research Allocation, 371,800 CPU hours. Second semester, 2021--22.
\item UChicago Midway2 Research Allocation, 352,00 CPU hours. First semester, 2021--22.
\item XSEDE Startup Allocation, 200,000 CPU hours (TG-AST190038, 2019). Highly-detailed strong-gravitational lens modeling to measure the Hubble constant.   
\end{enumerate}

\section{\sc Approved Observing Proposals/Grants (CoI)}
\begin{enumerate}
\item Nancy Grace Roman Space Telescope Research and Support Participation Opportunities (2023). PI: Pierel. Enhancing the Roman Cosmology Program with Strongly Lensed Supernovae.
\item \textit{Hubble Space Telescope} GO-17474 (2023). PI: Pierel. Pioneering Precision: Advancing Cosmology with the First Statistical Sample of Gravitationally Lensed Supernovae
\item \textit{Hubble Space Telescope} GO-17437 (2023). PI: Barone. When does the initial mass function become heavy? A unique view of two massive galaxies at $z=1$.
\item \textit{Hubble Space Telescope} GO-17130 (2022). A 4\% determination of the Hubble constant from gravitational time delays with maximally flexible lens mass profile. \textbf{Grant managed by Shajib: $\sim$\$40K}.
\item Very Large Telescope, MUSE, P110 (2022). PI: Zanella. From cosmology to star-forming regions: two compelling cases for MUSE narrow-field mode.
\item \textit{James Webb Space Telescope} GO-1794 (Cycle 1, 2021). PI: Suyu. 100\% gain in precision and accuracy of H0 measurement from JWST stellar kinematics of a lens galaxy. 
\item Very Large Telescope, MUSE, P108 (2021). PI: Zanella. From cosmology to star-forming regions: two compelling cases for MUSE narrow-field mode.
\item \textit{Hubble Space Telescope} GO-16773 (2021). PI: Glazebrook. A SNAPshot Legacy Survey of Bright Gravitational Lenses. \textbf{Grant managed by Shajib: $\sim$\$120K}.
\item \textit{Hubble Space Telescope} GO-15652 (2018). PI: Treu. $H_0$, the stellar initial mass function, and other dark matters from a large sample of quadruply imaged quasars.
\item 2-m Himalayan Chandra Telescope (2018). PI: Courbin. Photometric monitoring of the quadruply lensed quasar PSOJ0147+4630.
\item Very Large Telescope, MUSE NFM Science Verification (2018, 103A). PI: Zanella. From cosmology to star-forming regions: two compelling cases for MUSE NFM.
\item Keck Observatory, U053(2017A), U032(2017B), U011(2018A),  U011(2018B), U029(2019A), U065(2019B), U021(2021A), U030(2022B), U028(2023B). PI: Treu.
\item Subaru Telescope, S21A-0128N, S22B-0102N, S23B-0057N. PI: Wong.
\item Gemini South Telescope, 2022B-Q-230, 2023A, 2023B-986125. PI: Buckley-Geer.
\end{enumerate}

\section{\sc Collaboration Membership}
\begin{itemize}
	\item Rubin Observatory LSST's Dark Energy Science Consortium (DESC), \textbf{co-convener (April 2023--present)} of Strong-Lensing Topical Team (SLTT) 
	\item STRong-lensing Insights into Dark Energy Survey (STRIDES), an external collaboration of the Dark Energy Survey (DES), \textbf{Co-PI}
	\item Time-delay Cosmography (TDCOSMO), \textbf{co-chair} of environment analysis subgroup, 2021--2023.
	\item $H_0$ Lenses in COSMOGRAIL's Wellspring (H0LiCOW)
	\item Rubin Observatory LSST's Strong Lensing Science Consortium (SLSC)
	\item \href{https://www.lenswatch.org/}{LensWatch}
\end{itemize}

\section{\sc Professional Service}
\begin{itemize}
\item TAC member for NASA, 2023--2024
\item SOC member, NHFP symposium, 2022
\item Subject-matter expert reviewer in a NASA peer review, 2022
\item Future Leader participant, AURA annual meeting, 2021
\item Referee for MNRAS (Monthly Notices of the Royal Astronomical Society) and ApJ (The Astrophysical Journal, American Astronomical Society)
%\item Proposal reviewer for \textit{Hubble Space Telescope}
\item Graduate admission committee member (2019), Division of Astronomy, UCLA
\end{itemize}

\section{\sc Conference organizing}
\begin{enumerate}
	\item SOC chair of KICP workshop. Lensing at Different Scales: strong, weak, and synergies between the two, 2023. \underline{\$15K grant} from KICP, UChicago.
\end{enumerate}

\section{\sc {\underline{Invited}} conference talks}
\begin{enumerate}
	\item IAUS 381: Strong gravitational lensing in the era of big data. Italy, June 2023.
	\item Workshop on Bridging Gaps between Dynamical Probes of Galaxies, Lorentz Center, Netherlands, April 2022.
	\item Dark Energy Survey meeting plenary talk, May 2020.	 %remote
\end{enumerate}

\section{\sc Colloquia and Seminar Talks\\ ({*}\ invited)}
\begin{enumerate}
	\item *Seminar, Dept. of Physics \& Astronomy, University of California, Davis, May 2023.
	\item *Seminar, Dept. of Astronomy, Boston University, March 2023.
	\item *Seminar, Dept. of Physical Sciences, Independent University, Bangladesh, March 2023.
	\item {*}Seminar, Dept. of Physics \& Astronomy, Johns Hopkins University, February 2023.
	\item {*}Astrophysics Symposium, Physics Department, Yale University, USA, January 2022.
	\item {*}Astronomy Colloquium, Indiana University, Bloomington, USA, January 2022.
	\item Virtual Astronomy Software Talk (VAST), virtual, December 2022.
	\item {*}Astronomy Lunch Seminar (virtual), Kavli IPMU, University of Tokyo, Japan, November 2022.
	\item {*}KICC Cosmology Group Seminar, Cambridge University, UK, March 2022.
	\item {*}Physics Web-colloquium Series, Pabna University of Science and Technology, Bangladesh, July 2021.
	\item {*}Argonne National Lab--UChicago Joint Cosmology Meeting, USA, May 2021.
	\item {*}Open seminar by Dvorkin Group, Harvard University, USA, April 2021.
	\item Survey Science Group Meeting, University of Chicago, USA, October 2020.
	\item {*}Colloquium (remote), Institute of Cosmology and Gravitation, University of Portsmouth, UK, April 2020.
	\item FLASH Friday talk, University of California, Santa Cruz, California, USA, December 2019.
	\item Science talk, Infrared Processing and Analysis Center, California Institute of Technology, USA, November 2019.
	\item Astrophysics Seminar, Jet Propulsion Laboratory, California, USA, November 2019.
	\item Cosmology seminar, Berkeley Center For Cosmological Physics, University of California, Berkeley, USA, November 2019. 
	\item *Cosmology seminar, Kavli Institute for Particle Astrophysics and Cosmology, Stanford University, USA, November 2019.
	\item Journal club seminar, Center for Astrophysics and Space Sciences, University of California, San Diego, USA, November 2019.
	\item Astrophysics seminar, University of California, Irvine, USA, October 2019.
	\item Thursday lunch seminar, Princeton University, USA, October 2019.
	\item Galaxy lunch talk, Yale University, USA, October 2019.
	\item Galaxies and Cosmology Seminar, Center for Astrophysics, Harvard \& Smithsonian, USA, October 2019.
	\item Galaxy journal club, Space Telescope Science Institute, USA, October 2019.
	\item Particle Astrophysics Seminar, Fermilab, USA, October 2019.
	\item Lunch talk, Carnegie Observatories, Pasadena, USA, September 2019.
	\item Astronomy seminar. University of California, Riverside, USA, May 2019. 
	\item {*}MPA Lensing Group Seminar, Munich, Germany, June 2018.
\end{enumerate}


\section{\sc Contributed Talks}
\begin{enumerate}
	\item NHFP Fellows' Symposium, Center for Astrophysics, Harvard \& Smithsonian, Boston, USA, September 2023.
	\item The Extragalactic Distance Scale and Cosmic Expansion in the Era of Large Surveys and the JWST. MIAPbP workshop, Garching, Germany, July 2023. 
	\item 241\textsuperscript{st} American Astronomical Society Meeting, USA, January 2023.
	\item NHFP Fellows' Symposium. STScI, Baltimore, USA, September 2022.
	\item Boom! workshop on Explosive Transients with LSST. The University of Illinois at Urbana-Champaign, USA, July 2022.
	\item NHFP Fellows' Symposium, remote, October 2021.
	\item Spatially Resolved Spectroscopy with Extremely Large Telescopes, University of Oxford, UK, September 2021. %virtual
	\item 237\textsuperscript{th} American Astronomical Society Meeting, USA, January 2021. % virtual
	\item Dissertation talk, 235\textsuperscript{th} American Astronomical Society Meeting, Hawaii, USA, January 2020.
	\item Dark Energy Survey meeting, University of Sussex, UK, November 2019. % remote
	\item Non-Standard Cosmology Probes, Aspen Center of Physics workshop, Colorado, USA, August 2019.
	\item Tensions between the Early and the Late Universe. Kavli Institute for Theoretical Physics, University of California, Santa Barbara, USA, July 2019.
	\item Keck Science Meeting. Caltech, USA, September 2018.
	\item Extragalactic distance scale in the \textit{GAIA} era, Munich Institute for Astro- and Particle Physics workshop. Germany, June 2018.
	\item Shedding Light on the Dark Universe with Extremely Large Telescopes. UCLA, USA, April 2018.
	\item Strong Lensing by Galaxies and Clusters. Aosta, Italy, June 2017.
\end{enumerate}


\section{\sc Workshops \\ ({*}\ invited)}
\begin{enumerate}
	\item The Extragalactic Distance Scale and Cosmic Expansion in the Era of Large Surveys and the JWST. MIAPbP, Garching, Germany, July 2023.
	\item *International Space Science Institute (ISSI) workshop on Strong Lensing, Switzerland, July 2022.
	\item *Bridging Gaps between Dynamical Probes of Galaxies, Lorentz Center, Netherlands, April 2022.
	\item Non-Standard Cosmology Probes, Aspen Center of Physics, Colorado, USA, August--September 2019.
	\item TMT Early Career Initiative Workshop, Los Angeles, December 2018.
	\item Extragalactic distance scale in the \textit{GAIA} era, MIAPP, Garching, Germany, June--July 2018.
	\item Mary Lea \& C. Donald Shane Observational Astronomy Workshop, UCO/Lick Observatory, October 2014.
\end{enumerate}


\section{\sc Media Coverage}
\begin{enumerate}
	\item Siegel, E., ``Astronomically Rare `Double Lens' Yields Best Single System Measurement Of Cosmic Expansion'', \href{https://www.forbes.com/sites/startswithabang/2019/10/28/astronomically-rare-double-lens-yields-best-single-system-measurement-of-cosmic-expansion/#55acbc504373}{Forbes, 2019}.	
\end{enumerate}


\section{\sc Poster presentation}
\begin{enumerate}
	\item Cosmic Controversies. Kavli Institute for Cosmological Physics, University of Chicago, USA, October 2019.
	\item Tensions between the Early and the Late Universe. Kavli Institute for Theoretical Physics, University of California, Santa Barbara, USA, August 2019.
\end{enumerate}


\section{\sc Observing Experience}
OSIRIS, Keck I, 18.5 nights,\\
NIRC2, Keck II, 3 nights, \\
MOSFIRE, Keck I, 3 nights, \\
Shane telescope PFcam and Nickel telescope imager, Lick Observatory, 1 night.


\section{\sc Data Analysis Experience}
\textit{Hubble Space Telescope} (WFC3), 
W. M. Keck Observatory (OSIRIS, NIRC2),
Very Large Telescope (MUSE),
{\it Wide-field Infrared Survey Explorer},
{\it Wilkinson Microwave Anisotropy Probe},
{\it Planck},
Sloan Digital Sky Survey.

\section{\sc Scientific Software Development} 
\begin{itemize}
	\item Lead developer of lens-modeling automator \textsc{dolphin} \href{https://github.com/ajshajib/dolphin}{\faGithub}.
	\item Co-developer and maintainer for the lens-modeling software \textsc{lenstronomy} \href{https://github.com/lenstronomy/lenstronomy}{\faGithub}, an \href{https://www.astropy.org/affiliated/index.html}{affiliated package} of \textsc{Astropy}.
\end{itemize}

\section{\sc Computer Skills} 
%\begin{list2}
\textbf{Programming Languages:} Python, C, C++, PHP, SQL, JavaScript \\
\textbf{Other Software/Framework:} TensorFlow, Flask


\section{\sc Mentoring}
\begin{itemize}
	\item \textbf{Eden Molina:} UCLA undergraduate student, completed a project to model doubly-imaged lensed quasars from NIRC2 imaging data. Mentored Fall 2018--Winter 2020. Coauthored and published a paper (\href{https://doi:10.1093/mnras/stab532}{Shajib, Molina, et al., 2021}).
	\item \textbf{Vedant Sahu:} UCLA undergraduate student, worked on a project to apply machine learning techniques in modeling quadruply-lensed quasars. Mentored Summer 2019--Spring 2021.
	\item \textbf{Chin Yi Tan:} UChicago graduate student, working on a project to build an automated pipeline for modeling galaxy--galaxy lenses. Mentored since Winter 2021. \textit{Supported through an \textsc{HST} grant with myself being the PI.}
	\item \textbf{Hannah Skobe:} UChicago post-baccalaureate scholar, working on a project to upscale lower-resolution astronomical images using machine learning. Mentored since Spring 2021. \textit{Supported through an HST grant with myself being the grant PI.} Went on to the PhD program at CMU.
	\item \textbf{Abigail Lee:} UChicago graduate student, worked on measuring cosmological parameters from a time-delay strong lensing system. Mentored Summer 2021--Summer 2022.
	\item \textbf{Aidan Cloonan:} UChicago undergraduate student, working on a project to compare structural properties of strong lensing galaxies and the parent population of elliptical galaxies. Mentored since Summer 2021. Went on to the PhD program at UMass Amherst.
	\item \textbf{Pierre Boccard:} EPFL master's student on an exchange program to UChicago, worked on and defended master's thesis to measure the dark energy parameter $w$ using a compound lens system. Mentored since Winter 2023.
	\item \textbf{Xianzhe Tang:} Stony Brook University undergraduate student. Co-mentored with Prof. Simon Birrer. Worked on including line-of-sight structures in the strong lensing simulation pipeline for the Rubin Observatory's LSST. Summer 2023.
\end{itemize}


\section{\sc Teaching}

{\bf University of Illinois, Chicago}, USA

{\em Guest Lecturer} \hfill {\bf}\\
\begin{list2}
	\item GE course on Astronomy \& Universe (Spring 2023, Prof.~Cecilia Gerber)
\end{list2}

{\bf University of Chicago}, USA

{\em Guest Lecturer} \hfill {\bf}\\
\begin{list2}
	\item Astro 285 - Science with Large Astronomical Surveys (Spring 2023, Prof.~Alex Drlica-Wagner)
	\item Astro 298 - Undergraduate Research Seminar (Spring 2022, Prof.~Hsiao-Wen Chen)
 	\item Graduate course on Gravitational Lensing (Fall 2020, Prof.~Chihway Chang)
\end{list2}
	
	
{\bf University of California, Los Angeles}, USA

%\vspace{-.3cm}
%{\em Graduate Student} \hfill {\bf October 2014 - present}\\
%Includes current Ph.D.~research, Ph.D.~and Masters level coursework and
%research.


{\em Guest Lecturer} \hfill {\bf}\\
\begin{list2}
	\item Physics 127 - General Relativity (Spring 2015, Dr.~Slava Turyshev)
	\item Astro 81 - Astronomy I: Stars and Nebulae (Winter 2016, Prof.~Andrea Ghez)
\end{list2}
	
{\em Teaching Assistant} \hfill {}\\
\begin{list2}
	\item Astronomy 3 - Nature of Universe (Fall 2014)
	\item Physics 1C - Electrodynamics, Optics and Special Relativity (Winter 2015)
	\item Physics 127 - General Relativity (Spring 2015)
	\item Physics 6C - Physics for Life Sciences Majors: Light, Fluids, Thermodynamics, Modern Physics (Fall 2015)
	\item Astronomy 81 - Astrophysics I: Stars and Nebulae (Winter 2016)
	\item Astronomy 140 - Stellar Systems and Cosmology (Spring 2016)
	\item Physics 12 - Physics of Sustainable Energy (Winter 2017)
\end{list2}

%{\bf European Southern Observatory}, Munich, Germany\\
%\vspace{-.1cm}
%{\em Visiting Graduate Student} \hfill {\bf July 2018}\\
%Collaborative research with Dr. Adriano Agnello.

%\vspace{-.1cm}
%{\em Instructor} \hfill {\bf May - June, 2002}\\
%Co-taught graduate level course for the Master of Science in
%Computational Finance program.  Shared responsibility for lectures, exams,
%homework assignments, and grades.  
%\vspace*{.05in}  
%\begin{list2}
%\item 46-731 Probability and Statistics, Summer 2002.
%\end{list2}


%\vspace{-.1cm}
%{\em NSF VIGRE Teaching Fellow} \hfill {\bf January - May, 2001}\\
%Head teaching assistant.   
%Duties included shared administrative responsibilities with faculty
%instructor, fielding all student inquiries, and oversight of
%graduate student teaching assistants and graders.
%\vspace*{.05in}  
%\begin{list2}
%\item 36-217 Probability Theory and Random Processes, Spring 2001.
%\end{list2}


%\vspace{-.1cm}

%Paciorek, C.J., J.S. Risbey, V. Ventura, and R.D.Rosen.  2001.  Changes in Northern Hemisphere winter storm activity (1949-1999) based
%on a comparison of cyclone indices.  8th International Meeting on
%Statistical Climatology, Luneberg, Germany, March 2001.
%
%Paciorek, C.J. and R. Rosenfeld.  2000.  Minimum classification error
%training in exponential language models.  2000 Spring Transcription
%Workshop, College Park, Maryland.
%\vspace*{-.25in}  
%\begin{verbatim}http://www.nist.gov/speech/publications/tw00/html/abstract.htm#cp1-50\end{verbatim}

%\section{\sc Professional Experience}
%{\bf Bureau of Transportation Statistics, U.S. Department of
%  Transportation}, Washington, District of Columbia USA
%
%\vspace{-.3cm}
%{\em Summer researcher} \hfill {\bf May, 2000 - August, 2000}\\
%Carried out several consulting projects, including the modeling of
%injuries to cadavers in crash test experiments, analysis of airline
%delay data, and advice on analysis of airline economics data.
%
%{\bf Abt Associates}, Bethesda, Maryland USA
%
%\vspace{-.3cm}
%{\em Associate Programmer Analyst and Research Assistant} \hfill {\bf
%  October 1994 - August 1996}\\
%Researcher and computer model developer for U.S. EPA Regulatory Impact
%Analysis of Section 403 Lead Paint Hazard Rule.  Other projects
%included database analysis, literature reviews, and cost-benefit analysis.


\section{\sc Inclusion \& Access}
{\bf Founder and Coordinator} of \href{https://www.astrobridge.org/}{Astro Bridge}, a bridge program for undergraduate students from countries lacking research opportunities at the undergraduate level. \\
{\bf Mentor} of an Astronomy research workshop for Bangladeshi undergraduate/master's students with 35 participants, under the \href{https://www.astrobridge.org/projects/bdlensing}{Astro Bridge} program. February 2023--present. \\ 

\section{\sc Outreach}
{\bf Public talk}, Shahjalal University of Science \& Technology, Bangladesh, March 2023. \\
{\bf Coordinator} of {\bf Lifelong Learning Outreach program}, KICP, 2022--23. \\
{\bf Speaker} at {\bf Lifelong Learning Talk series}, multiple talks at the Chicago Public Library and senior centers, 2022. \\
{\bf Cal-Bridge program}, hosted a workshop at UCLA for California State University undergraduates on Graduate admission preparation, March 2019. \\
{\bf Lecturer at Astronomy Live! Summer workshop} for high school students, 2018. \\
{\bf Astronomy Live!}, visited K-12 schools to perform various demos as part of the UCLA Astronomy outreach program. \\
{\bf Exploring Your Universe}, performed various demos in UCLA's annual science festival, 2014-17.
{\bf Star show presenter}, UCLA Planetarium, 2014-2016. \\
{\bf Public talk}, UCLA Planetarium, 2014. \\


%\clearpage
%\iffalse

\section{\sc Publications}
\textbf{First-author publications}
\newcommand{\mentee}{${\boldsymbol{\dagger}}$}
\\ \mentee\ Mentee
\begin{enumerate}
	\item \textbf{Shajib, A.~J.}, et al. TDCOSMO. XII. Improved Hubble constant measurement from lensing time delays using spatially resolved stellar kinematics of the lens galaxy. \href{https://ui.adsabs.harvard.edu/abs/2023arXiv230102656S/abstract}{A\&A, 673, A9, 2023}.
	\item \textbf{Shajib, A.~J.}, et al. Strong Lensing by Galaxies. Invited review article for ISSI workshop on strong lensing, to be submitted to Space Science Reviews. \href{https://arxiv.org/abs/2210.10790}{arXiv:2210.10790, 2022}.
	\item \textbf{Shajib, A.~J.}, et al. LensingETC: a tool to optimize multi-filter imaging campaigns of galaxy-scale strong lensing systems. \href{https://doi.org/10.3847/1538-4357/ac927b}{ApJ, 938, 141, 2022}.
	\item \textbf{Shajib, A.~J.}, et al. TDCOSMO. IX. Systematic comparison between lens modelling software programs: time delay prediction for WGD 2038$-$4008. \href{https://arxiv.org/abs/2202.11101}{A\&A, 667, A123, 2022}.
	\item \textbf{Shajib, A.~J.}, et al. Dark matter haloes of massive elliptical galaxies at z $\sim$ 0.2 are well described by the Navarro--Frenk--White profile. \href{https://doi.org/10.1093/mnras/stab536}{MNRAS, 503, 2, 2380-2405, 2021}.
	\item \textbf{Shajib, A.~J.}, Molina, E.{\mentee}, et al. High-resolution imaging follow-up of doubly imaged quasars. \href{https://doi:10.1093/mnras/stab532}{MNRAS, 503, 2, 1557-1567, 2021}.
	\item \textbf{Shajib, A.~J.}, et al. STRIDES: A 3.9 per cent measurement of the Hubble constant from the strong lens system DES J0408--5354. \href{https://academic.oup.com/mnras/advance-article-abstract/doi/10.1093/mnras/staa828/5813265}{MNRAS, 494, 6072--6102, 2020}.
	\item \textbf{Shajib, A.~J.} Unified lensing and kinematic analysis for \textit{any} elliptical mass profile. \\ \href{https://doi.org/10.1093/mnras/stz1796}{MNRAS, 488, 1387--1400, 2019}.
	\item \textbf{Shajib, A.~J.}, et al. Is every strong lens model unhappy in its own way? Uniform modelling of a sample of 13 quadruply+ imaged quasars. \href{https://doi.org/10.1093/mnras/sty3397}{MNRAS, 483, 5649--5671, 2019}.
	\item \textbf{Shajib, A.~J.}, Treu, T., and Agnello, A. Improving time-delay cosmography with spatially resolved kinematics. \href{https://doi.org/10.1093/mnras/stx2302}{MNRAS, 473, 210--226, 2018}.
	\item \textbf{Shajib, A.~J.} and Wright, E.~L. Measurement of the integrated Sachs-Wolfe effect using the AllWISE data release. \href{http://dx.doi.org/10.3847/0004-637X/827/2/116}{ApJ, 827:116 (9pp), 2016}.
\end{enumerate}


\textbf{Second-author publications}
\\ \mentee\ Mentee
\begin{enumerate}
	\item Tan, C. Y.{\mentee}, \textbf{Shajib, A.~J.}, et al. Project Dinos I: A joint lensing--dynamics constraint on the deviation from the power law in the mass profile of massive ellipticals. In preparation, to be submitted in November 2023.
	\item Treu, T. and \textbf{Shajib, A.~J.} Strong Lensing and $H_0$. \href{https://arxiv.org/abs/2307.05714}{arXiv:2307.05714, 2023}.
	\item Birrer, S., \textbf{Shajib, A.~J.}, et al. lenstronomy II: A gravitational lensing software ecosystem. \href{https://joss.theoj.org/papers/10.21105/joss.03283}{Journal of Open Source Software, 6(62), 3283, 2021}.
	\item Birrer, S., \textbf{Shajib, A. J.}, et al. TDCOSMO IV: Hierarchical time-delay cosmography -- joint inference of the Hubble constant and galaxy density profiles. \href{https://doi.org/10.1051/0004-6361/202038861}{A\&A 643, A165, 2020}.
\end{enumerate}

\newpage
\textbf{Other co-authored publications}
\begin{enumerate}
	\item Gomer, M. R., et al. Ellipticity parameterization for an NFW profile: an overlooked angular structure in strong lens modeling. \href{https://arxiv.org/abs/2310.03077}{arXiv:2310.03077, 2023}.
	\item Sonnenfeld, A., et al. Strong lensing selection effects. \href{https://arxiv.org/abs/2301.13230}{arXiv:2301.13230, 2023}.
	\item Pierel, J.~D.~R., et al. LensWatch: I. Resolved HST Observations and Constraints on the Strongly-Lensed Type Ia Supernova 2022qmx (``SN Zwicky'').  \href{https://arxiv.org/abs/2211.03772}{arXiv:2211.03772, 2022}.
	\item Zaborowski, E., et al. Identification of Galaxy-Galaxy Strong Lens Candidates in the DECam Local Volume Exploration Survey Using Machine Learning. \href{https://arxiv.org/abs/2210.10802}{arXiv:2210.10802, 2022}.
	\item Birrer,~S., Millon,~M., Sluse,~D., \textbf{Shajib,~A.,} et al. Time-Delay Cosmography: Measuring the Hubble Constant and other cosmological parameters with strong gravitational lensing. \href{https://arxiv.org/abs/2210.10833}{arXiv:2210.10833, 2022}.
	\item Mozumdar, P., et al. TDCOSMO. XII. New lensing galaxy redshift and velocity dispersion measurements from Keck spectroscopy of eight lensed quasar systems. \href{https://arxiv.org/abs/2209.14320}{arXiv:2209.14320, 2022}.
	\item Ertl, S., et al. TDCOSMO XI. Automated Modeling of 9 Strongly Lensed Quasars and Comparison Between Lens Modeling Software. \href{https://arxiv.org/abs/2209.03094}{arXiv:2209.03094, 2022}.
	\item Lemon, C., et al. Gravitationally lensed quasars in Gaia -- IV. 150 new lenses, quasar pairs, and projected quasars. \href{https://arxiv.org/abs/2206.07714}{arXiv:2206.07714, 2022}.
	\item Schmidt,~T., Treu, T., Birrer, S., \textbf{Shajib, A.~J.}, et al. STRIDES: Automated uniform models for 30 quadruply imaged quasars. \href{https://arxiv.org/abs/2206.04696}{arXiv:2206.04696, 2022}.
	\item Morgan,~R., et al. DeepZipper II: Searching for Lensed Supernovae in Dark Energy Survey Data with Deep Learning. \href{https://arxiv.org/abs/2204.05924}{arXiv:2204.05924, 2022}.
	\item Akhazhanov, A., et al. Finding quadruply imaged quasars with machine learning. I. Methods. \href{https://ui.adsabs.harvard.edu/abs/2022MNRAS.tmp..904A/abstract}{MNRAS, 513, 2, 2407-2421, 2022}.
	\item Birrer, S., Dhawan. S., and \textbf{Shajib, A.~J.} The Hubble constant from strongly lensed supernovae with standardizable magnifications. \href{https://iopscience.iop.org/article/10.3847/1538-4357/ac323a}{ApJ, 924, 1, 2, 2022}.
	\item Ding, X., et al. Time Delay Lens Modelling Challenge. \href{https://ui.adsabs.harvard.edu/abs/2021MNRAS.503.1096D/abstract}{MNRAS, 503, 1096-1123, 2021}.
	\item Buckley-Geer, E.~J., et al. STRIDES: Spectroscopic and photometric characterization of the environment and effects of mass along the line of sight to the gravitational lenses DES J0408$-$5354 and WGD 2038$-$4008. \href{https://ui.adsabs.harvard.edu/abs/2020MNRAS.498.3241B/abstract}{MNRAS, 498, 3, 3241-3274, 2020}.
	\item Lemon, C., et al. The STRong lensing Insights into the Dark Energy Survey (STRIDES) 2017/2018 follow-up campaign: Discovery of 10 lensed quasars and 10 quasar pairs. \href{https://doi.org/10.1093/mnras/staa652}{MNRAS, 494, 3, 3491-3511, 2020}.
	\item Millon, M., et al. TDCOSMO - I. An exploration of systematic uncertainties in the inference of $H_0$ from time-delay cosmography. \href{https://doi.org/10.1051/0004-6361/201937351}{A\&A, 639, A101, July 2020}.
	\item Wong, C.~K., et al. H0LiCOW – XIII. A 2.4 per cent measurement of $H_0$ from lensed quasars: 5.3$\sigma$ tension between early- and late-Universe probes. In press (MNRAS), \href{https://doi.org/10.1093/mnras/stz3094}{MNRAS, 498, 1, 1420-1439, 2020}.
	\item Chen, G.~C.-F., et al. A SHARP view of H0LiCOW: $H_0$ from three time-delay gravitational lens systems with adaptive optics imaging. \href{https://academic.oup.com/mnras/article/doi/10.1093/mnras/stz2547/5568378/}{MNRAS, 490, 1743--1773, 2019}.
	\item Taubenberger, S., et al. The Hubble Constant determined through an inverse distance ladder including quasar time delays and Type Ia supernovae. \href{https://www.aanda.org/articles/aa/abs/2019/08/aa35980-19/aa35980-19.html}{A\&A, 628, L7, 2019}.
	\item Rusu, C.~E., et al. H0LiCOW XII. Lens mass model of WFI2033-4723 and blind measurement of its time-delay distance and $H_0$. \href{http://adsabs.harvard.edu/abs/2019arXiv190509338R}{MNRAS, 498, 1, 2020, 1420-1439, 2020}.
	\item Sluse, D., et al. H0LiCOW X: Spectroscopic/imaging survey and galaxy-group identification around the strong gravitational lens system WFI2033-4723. \href{https://academic.oup.com/mnras/article/doi/10.1093/mnras/stz2483/5561514/}{MNRAS, 490, 613--633, 2019}.
	\item Birrer, S., et al. H0LiCOW - IX. Cosmographic analysis of the doubly imaged quasar SDSS 1206+4332 and a new measurement of the Hubble constant. \href{https://doi.org/10.1093/mnras/stz200}{MNRAS, 484, 4726--4753, 2019}.
	\item Chen, G.~C.-F., et al. Constraining the microlensing effect on time delays with new time-delay prediction model in $H_0$ measurements. \href{https://doi.org/10.1093/mnras/sty2350}{MNRAS, 481, 1115--1125, 2018}.
 	\item Williams, P.~R., et al. Discovery of three strongly lensed quasars in the Sloan Digital Sky Survey. \href{https://doi.org/10.1093/mnrasl/sly043}{MNRAS: Letters, 477, L70--L74, 2018}.
	%\item Molina, E., et al. More massive galaxies are more massive: luminous and dark matter in small-separation quasar lenses. In preparation.
\end{enumerate}
%\section{\sc Papers in preparation}
%\textbf{Shajib, A.J.} et al. Improving time-delay cosmography with spatially resolved kinematics. Submitted to MNRAS, 2017.

\textbf{Non-refereed papers}
\begin{enumerate}
	\item \textbf{Shajib, A.~J}. Strong lensing by galaxies: past highlights, current status, and future prospects. Proceeding of IAU symposium 381, 2023. \href{https://arxiv.org/abs/2310.07695}{doi:10.1017/S1743921323003903}.
	\item Tan, C.~Y.\mentee~\& \textbf{Shajib, A.~J}. Joint lensing--dynamics constraint on the elliptical galaxy mass profile from the largest galaxy--galaxy lens sample. Proceeding of IAU symposium 381, 2023.
	\item Di Valentino, E., et al. Snowmass2021 - Letter of interest cosmology intertwined IV: The age of the universe and its curvature. \href{https://www.sciencedirect.com/science/article/abs/pii/S0927650521000517}{Astroparticle Physics, Volume 131, 102607, 2021}.
	\item Di Valentino, E., et al. Snowmass2021 - Letter of interest cosmology intertwined III: $f\sigma_8$ and $S_8$. \href{https://www.sciencedirect.com/science/article/abs/pii/S0927650521000487}{Astroparticle Physics, Volume 131, 102604, 2021}.
	\item Di Valentino, E., et al. Snowmass2021 - Letter of interest cosmology intertwined II: The Hubble constant tension. \href{https://www.sciencedirect.com/science/article/abs/pii/S0927650521000499}{Astroparticle Physics, Volume 131, 102605, 2021}.
	\item Di Valentino, E., et al. Snowmass2021 - Letter of interest cosmology intertwined I: Perspectives for the next decade \href{https://www.sciencedirect.com/science/article/abs/pii/S0927650521000505}{Astroparticle Physics, Volume 131, 102606, 2021}.
	\item Beaton, R. L., et al. Measuring the Hubble Constant Near and Far in the Era of ELT's. \href{https://ui.adsabs.harvard.edu/abs/2019BAAS...51c.456B/abstract}{BAAS 51(3) 456, 	2019}.
	\item Ding, X., Treu, T., {\bf Shajib, A. J.}, et al. Time Delay Lens Modelling Challenge: I. Experimental Design. \href{https://arxiv.org/abs/1801.01506}{arXiv:1801.01506, 2018}.

\end{enumerate}

%\fi


%\end{list2}


%\section{\sc Positions of Responsibility}
%{\bf Captain and Coach,} The University of Tokyo Cricket Club, 2012-13 \\
%{\bf College prefect,} Sylhet Cadet College, 2006-07


\end{resume}
\end{document}

